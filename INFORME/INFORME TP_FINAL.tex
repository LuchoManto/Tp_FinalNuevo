%--------------------------------------------------------------
%     Ejemplo de un artículo con LaTeX
%--------------------------------------------------------------
\documentclass[a4paper]{article}
\usepackage[utf8]{inputenc}
\usepackage[spanish]{babel}
\usepackage{amssymb}
\usepackage{amsthm}
\usepackage{amsmath}

%para usar hipertexto en las referencias
\usepackage{hyperref}
\hypersetup{
    colorlinks,
    citecolor=black,
    filecolor=black,
    linkcolor=black,
    urlcolor=black
}
%para incluir graficos
\usepackage{graphicx}
%para permitir que la imagen este donde yo quiero y no en otro lugar
\usepackage{float}

\usepackage{fancyhdr}

\pagestyle{fancy}
\fancyhf{}
\rhead{Año 2015}
\lhead{Trabajo Final}
\rfoot{Pagina \thepage}
\lfoot{Electronica Digital III}

\graphicspath{ {Imagenes/} }


%--------------------------------------------------------------
\title{\underline{Electronica Digital III} \\
\huge \textbf{ \\ Trabajo Final} \\ }
\author{Autores: Ignacio Sambataro, Luciano Mantovani, Gaston Lucero Berrini\\ \\
  \large Profesores: Luis Alberto Murgio, Julio Sanchez \\
 % \small Facultad de Ciencias Exactas, Físicas y Naturales\\
 % \small Laboratorio de Arquitectura de Computadoras\\
 % \small Universidad Nacional de Cordoba\\
  \date{}
}
%--------------------------------------------------------------
\begin{document}
\thispagestyle{empty}
\maketitle
\thispagestyle{empty}
\begin{figure}[H]
\centering
\includegraphics[width=10cm, height = 3.5cm]{Escudo}
\end{figure}

\begin{center}
\small Facultad de Ciencias Exactas, Físicas y Naturales
\end{center}

\begin{center}
\small Laboratorio de Arquitectura de Computadoras
\end{center}

\begin{center}
\small Año 2015 \\
\end{center}

\pagebreak

\thispagestyle{empty}
%resumen
%aca hay que decir que basicamente es un trabajo final hecho para la catedra de digitales 3. lo hicimos con una lpc1769 y trabajamos haciendo funcionar todos los aspectos de un sensor de campo electrostatico, que requiere de un motor regulado a velocidades determinadas para funcionar. con las distintas funcionalidades de la LPC se pudo no solamente hacer andar el motor, sino tambien interpretar todos los datos que brinda, tanto de la velocidad del motor como las mediciones de campo electrico que responde
\abstract{Este documento es un informe sobre el trabajo final realizado para la catedra de Electronica Digital 3. Fue realizado bajo contexto academico con una placa de desarrollo LPC1769. El trabajo consistio en el funcionamiento y la adquisicion de datos de un sensor de campo electrostatico, que requirio de distintas funcionalidades de la placa de desarrollo para funcionar.}

%salto de página
\thispagestyle{empty}
\clearpage
\thispagestyle{empty}
\tableofcontents
\thispagestyle{empty}
\clearpage

\setcounter{page}{1}

%--------------------------------------------------------------

\section{Introduccion} % (fold)
\label{sec:introduccion}

%una breve introduccion del trabajo. cuales son los modulos de la LPC que usamos. como los usamos. 

En este trabajo hicimos un sistema que controla el funcionamiento y la adquisicion de datos de un sensor de campo electrostatico. La informacion de un sensor de este tipo se puede usar para determinar el nivel de electricidad estatica del ambiente producida por las nubes cercanas. Con esta informacion se puede determinar la posibilidad de que caiga un rayo en las cercanias del sensor.

Es necesario, para el funcionamiento del sensor, que se utilice un motor de continua o trifasico. Nuestro sistema se encarga de hacer funcionar este motor, controlar su velocidad, y obtener los datos del sensor via un conversor analogico digital. Todo esto fue obtenido mediante distintos modulos de la placa de desarrollo LPC1769. 

% section introduccion (end)

\section{Marco teorico} % (fold)
\label{sec:marco_teorico}



%como funciona un sensor de campo electrostatico? como funciona ESTE sensor? porque se necesita un motor para hacerlo funcionar? porque las distintas velocidades de un motor dan distintas respuestas? porque me interesa variar la velocidad del motor? 

%que detecta realmente el sensor? que hago con la informacion que me da?

El sensor de campo electrostatico es una estructura metalica compuesta por un motor que hace girar unas aspas que se se encargan de blindar y desblindar una placa que a su vez se carga y descarga con la electricidad estatica del ambiente. esta carga y descarga continua es lo que justamente se termina transformando en el nivel de voltaje que nos indica el nivel de electricidad estatica del ambiente, que es lo que queremos saber. Si imaginamos a la placa y a las aspas como un capacitor que se blinda y desblinda es mas facil. En el momento que las aspas estan descuburiendo la placa, el capacitor se carga, y en el momento que se cubre la placa, el capacitor se descarga. La descarga se hace sobre un amplificador que luego va a un conversor analogico digital, que termina en la lectura de un valor que nos dice el nivel del campo electrostatico ambiental.


\begin{figure}[h]
\centering
\includegraphics[width=10cm, height = 7cm]{sensor_1}
\caption{\small Vista frontal del sensor. Se pueden ver las aspas que, al girar, blindan y desblindan la placa conductora}
\label{fig:sensor_1}
\end{figure}


La intensidad de un campo electrico se puede medir, en principio, colcando un medidor de voltaje entre dos placas metalicas paralelas separadas por una distancia. El problema de esto es que, como el medidor de voltaje suele tener una impedancia alta en la entrada, cualquier voltaje inducido en las placas se pierde rapidamente, y no podria usarse para medir el capo electrico. Para arreglar esto, se utiliza la tecnica de las aspas. Se coloca una placa conductora, y sobre la misma se posiciona un sistema con aspas de forma que cuando estas roten, se cubra y se exponga periodicamente la placa conductora al campo electrico ambiental. Para lograr esto apropiadamente, el rotor que hace girar las aspas debe estar conectado a tierra. La placa conductora esta conectada a tierra a traves de un amplificador de transconductancia, que convierte la corriente que va desde la placa a tierra en una tension. A medida que la placa conductora este expuesta al campo electrico, el campo induce una corriente a tierra mientras que atrae o repele la carga de la placa conductora. A medida que la placa esta cubierta del campo electrico, la carga inducida se drena. Entonces las placas inducen una corriente alterna a masa que es proporcional a la intensidad del campo electrico estatico. Esta corriente alterna luego puede ser rectificada para utilizarla como entrada a un conversor analogico digital y obtener asi la intensidad del campo electrico medido.

\begin{figure}[H]
\centering
\includegraphics[width=6cm, height = 8cm]{sensor_2}
\caption{\small Vista trasera del sensor. Se puede ver el eje del motor que hace girar las aspas secundarias que permiten generar la señal en el optoacoplador que se usara para medir la velocidad del motor.}
\label{fig:sensor_2}
\end{figure}

Para asegurar la integridad de los datos medidos, es necesario que la velocidad de las aspas sea constante, ya que una variacion en la velocidad puede afectar el periodo con el que se mide el campo electrico y puede afectar directamente a la medicion. Un control minimo de velocidad sobre la velocidad del motor puede ser suficiente para que los datos leidos sean confiables. Para mantener esta velocidad constante, se utiliza un sistema de aspas secundarias que giran sobre el mismo eje del motor. A medida que gira el motor y giran las aspas secundarias, se permite y se bloquea el paso de luz de un fotoacoplador. Con la señal de este fotoacoplador se pueden determinar las vueltas por minuto de las aspas, y con esto verificar que la velocidad sea constante.



\begin{figure}[H]
\centering
\includegraphics[width=10cm, height = 7cm]{sensor_3}
\caption{\small Vista lateral del sensor}
\label{fig:sensor_3}
\end{figure}

% section marco_teorico (end)

\section{Desarrollo} % (fold)
\label{sec:desarrollo}

que hicimos? por donde arrancamos?


\subsection{Modulos Utilizados} % (fold)
\label{sub:modulos_utilizados}

% cuales son los modulos utilizados y porque? 

\subsubsection{Modulo PWM} % (fold)
\label{ssub:modulo_pwm}

breve descripcion del modulo pwm y como lo usamos

% subsubsection modulo_pwm (end)

\subsubsection{Modulo Timer} % (fold)
\label{ssub:modulo_timer}

breve descripcion del modulo timer y como lo usamos

% subsubsection modulo_timer (end)

\subsubsection{Modulo Conversor Analogico-Digital} % (fold)
\label{ssub:modulo_conversor_analogico_digital}

breve descripcion del modulo adc y como lo usamos

% subsubsection modulo_conversor_analogico_digital (end)

\subsubsection{Modulo de interrupciones externas y teclado matricial} % (fold)
\label{ssub:modulo_de_interrupciones_externas_y_teclado_matricial}

breve descripcion del modulo de interrupciones externas, del teclado matricial y como lo usamos

% subsubsection modulo_de_interrupciones_externas_y_teclado_matricial (end)

\subsubsection{Modulo UART} % (fold)
\label{ssub:modulo_uart}

breve descripcion del modulo UART y como lo usamos

% subsubsection modulo_uart (end)

% subsection modulos_utilizados (end)

\subsection{Comunicacion con Wi-Fi mediante servidor Web} % (fold)
\label{sub:comunicacion_con_wi_fi_mediante_servidor_web}

aca explicar todo lo hecho con python y el servidor web que se comunica con el modulo UART

% subsection comunicacion_con_wi_fi_mediante_servidor_web (end)

\section{Funcionamiento del programa} % (fold)
\label{sec:funcionamiento_del_programa}

Desde que se inicia, el sistema se configura automaticamente, pero necesita de interaccion humana para arrancar y medir tanto las RPM como las mediciones, o cambiar la velocidad del motor.

\subsection{Inicio} % (fold)
\label{sub:inicio}

En el inicio hace todas las configuraciones pertinentes, y despues queda en espera a una accion del usuatio

% subsection inicio (end)

\subsection{interaccion con el usuario} % (fold)
\label{sub:interaccion_con_el_usuario}

como hace el usuario para interactuar con el sistema?. lo puede hacer por teclado o mediante el servidor.. que permite esto?? acceso local y remoto.. etc

% subsection interaccion_con_el_usuario (end)

\subsection{Apagado} % (fold)
\label{sub:apagado}

como haces para apagar el motor? como haces para apagar el sistema?

% subsection apagado (end)

% section funcionamiento_del_programa (end)


\section{Conclusiones} % (fold)
\label{sec:conclusiones}

la fruta mas fruta de las frutas fruteras de la fruta

% section conclusiones (end)

% section desarrollo (end)







 \end{document} 